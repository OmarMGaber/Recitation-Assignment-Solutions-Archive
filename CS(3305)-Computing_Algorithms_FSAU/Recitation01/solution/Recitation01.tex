\documentclass{article}

\usepackage{listings}
\usepackage{geometry}
\geometry{left=1cm,right=1cm,top=1cm}

\title{Algorithms Recitation 01 Assignment}
\date{Date: \today}
\author{\Large Omar Muhammad Gaber \\ \Large 20221446137}

\lstset{
    breaklines=true
    columns=flexible,
}

\begin{document}

\maketitle

\section{Exercise 6: Palindrome Check}
\textbf{Problem Statement:}\\
Write a recursive method to check if a given string is a palindrome.\\\\
Sample Input: ” l e v e l ”\\
Expected Output: true

\begin{lstlisting}[language=C++, caption={Palindrome Check Subprogram}, label=cppcode]
bool isPalindromeHelper(const string &str, int left, int right) {
    if (left >= right)
        return true;
    if (str[left] != str[right])
        return false;
    return isPalindromeHelper(str, left + 1, right - 1);
}
bool isPalindrome(const string &str) {
    return isPalindromeHelper(str, 0, str.size() - 1);
}
\end{lstlisting}

\textbf{Time Complexity:} $O(n)$\\
\textbf{Auxiliary Space:} $O(n)$, where $n$ is the length of the string.\\\\

\section{Exercise 7: Array Sum}
\textbf{Problem Statement:}\\
Create a recursive method to find the sum of elements in an integer array.\\\\
Sample Input: [2, 4, 6, 8, 10]\\
Expected Output: 30

\begin{lstlisting}[language=C++, caption={Array Sum Subprogram}, label=cppcode]
template<typename T>
T ArraySum(vector<T> &arr, int length) {
    if (length == 0)
        return 0;
    return arr[length - 1] + ArraySum(arr, length - 1);
}
\end{lstlisting}

\textbf{Time Complexity:} $O(n)$\\
\textbf{Auxiliary Space:} $O(n)$, where $n$ is the size of the array.\\\\

\section{Exercise 8: Binary Search}
\textbf{Problem Statement:}\\
Implement a recursive method for binary search on a sorted array.\\\\
Sample Input: [1, 2, 3, 4, 5, 6, 7, 8, 9], target = 5\\
Expected Output: 4 (index of the target element)

\begin{lstlisting}[language=C++, caption={Binary Search Subprogram}, label=cppcode]
template<typename T>
int binarySearch(vector<T> &arr, T &val, int low, int high) {
    if (low > high)
        return -1;

    int mid = low + ((high - low) / 2);
    if (arr[mid] == val)
        return mid;
    if (arr[mid] > val)
        return binarySearch(arr, val, low, mid - 1);
    return binarySearch(arr, val, mid + 1, high);
}
\end{lstlisting}

\textbf{Time Complexity:} $O(\log n)$\\
\textbf{Auxiliary Space:} $O(\log n)$, where $n$ is the size of the array.\\\\

\section{Exercise 9: Reverse String}
\textbf{Problem Statement:}\\
Write a recursive method to reverse a given string.\\\\
Sample Input: "hello"\\
Expected Output: "olleh"

\begin{lstlisting}[language=C++, caption={Reverse String Procedure}, label=cppcode]
void swap(char &a, char &b) {
    char tempChar = a;
    a = b;
    b = tempChar;
}

void reverseString(string &str, int left, int right) {
    if (left >= right || right > str.length())
        return;

    swap(str[left], str[right]);
    reverseString(str, left + 1, right - 1);
}
\end{lstlisting}

\textbf{Time Complexity:} $O(n)$\\
\textbf{Auxiliary Space:} $O(n)$, where $n$ is the length of the string.\\\\

\newpage
\section{Exercise 10: Tower of Hanoi}
\textbf{Problem Statement:}\\
Implement the Tower of Hanoi problem using recursion.\\\\
Sample Input:\\
Number of disks = 3\\
Source = A\\
Auxiliary = B\\
Destination = C\\
Expected Output:\\
Move disk 1 from A to C\\
Move disk 2 from A to B\\
Move disk 1 from C to B\\
Move disk 3 from A to C\\
Move disk 1 from B to A\\
Move disk 2 from B to C\\
Move disk 1 from A to C

\begin{lstlisting}[language=C++, caption={Tower of Hanoi Procedure}, label=cppcode]
void hanoi(int n, const string &source, const string &auxiliary, const string &destination) {
    if (n == 1)
        cout << "Move disk " << n << " from " << source << " to " << destination << '\n';
    else {
        hanoi(n - 1, source, destination, auxiliary);
        cout << "Move disk " << n << " from " << source << " to " << destination << '\n';
        hanoi(n - 1, auxiliary, source, destination);
    }
}

int main() {
    int n;
    string source, auxiliary, destination;

    cout << "Number of disks: \n";
    cin >> n;

    cout << "Source: \n";
    cin >> source;

    cout << "Auxiliary: \n";
    cin >> auxiliary;

    cout << "Destination: \n";
    cin >> destination;

    hanoi(n, source, auxiliary, destination);

    return 0;
}
\end{lstlisting}
\textbf{Time Complexity:}
\[
T(n) = 2^{n} - 1
\]
\textbf{Auxiliary Space:} $O(n)$, where $n$ is the number of disks.\\\\

\section{Coin Changing Problem}
\textbf{Problem Statement:}\\
You are given an array representing different coin denominations and a total amount of money. The goal is to find the minimum number of coins needed to make up that amount. Assume an unlimited supply of coins of each denomination.\\

\textbf{Input:}
\begin{itemize}
    \item An array \texttt{coins} representing the coin denominations, where each coin denomination is a positive integer.
    \item An integer \texttt{amount} representing the total amount of money to make up.
\end{itemize}

\textbf{Output:}
\begin{itemize}
    \item An integer representing the minimum number of coins needed to make up the amount.
    \item If it's not possible to make up the amount using the given coin denominations, return -1.
\end{itemize}
\begin{lstlisting}[language=C++, caption={Coin Changing Problem Dynamic Programming Approach Subprogram}, label=cppcode]
int minCoins(vector<int>& coins, int target) {
    vector<int> dp(target + 1, INT_MAX);

    dp[0] = 0;

    for (int i = 1; i <= target; i++) {
        for (int coin : coins) {
            if (i - coin >= 0 && dp[i - coin] != INT_MAX) {
                dp[i] = min(dp[i], dp[i - coin] + 1);
            }
        }
    }

    return dp[target] == INT_MAX ? -1 : dp[target];
}
\end{lstlisting}
\textbf{Time Complexity:}
 $O(target * n)$\\
\textbf{Auxiliary Space:}
 $O(n)$, where $n$ is the number of coins.\\\\

\end{document}
